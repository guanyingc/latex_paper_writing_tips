% CVPR 2022 Paper Template
% based on the CVPR template provided by Ming-Ming Cheng (https://github.com/MCG-NKU/CVPR_Template)
% modified and extended by Stefan Roth (stefan.roth@NOSPAMtu-darmstadt.de)

\documentclass[10pt,twocolumn,letterpaper]{article}

%%%%%%%%% PAPER TYPE  - PLEASE UPDATE FOR FINAL VERSION
%\usepackage[review]{cvpr}      % To produce the REVIEW version
\usepackage{cvpr}              % To produce the CAMERA-READY version
%\usepackage[pagenumbers]{cvpr} % To force page numbers, e.g. for an arXiv version

% Include other packages here, before hyperref.
\usepackage{graphicx}
\usepackage{amsmath}
\usepackage{amssymb}
\usepackage{booktabs}

% included packages
\usepackage{makecell}
\usepackage{colortbl}
\usepackage{xcolor}

% It is strongly recommended to use hyperref, especially for the review version.
% hyperref with option pagebackref eases the reviewers' job.
% Please disable hyperref *only* if you encounter grave issues, e.g. with the
% file validation for the camera-ready version.
%
% If you comment hyperref and then uncomment it, you should delete
% ReviewTempalte.aux before re-running LaTeX.
% (Or just hit 'q' on the first LaTeX run, let it finish, and you
%  should be clear).
\usepackage[pagebackref,breaklinks,colorlinks]{hyperref}


% Support for easy cross-referencing
\usepackage[capitalize]{cleveref}
\crefname{section}{Sec.}{Secs.}
\Crefname{section}{Section}{Sections}
\Crefname{table}{Table}{Tables}
\crefname{table}{Tab.}{Tabs.}

% For caption setup
\captionsetup[table]{aboveskip=3pt}
\captionsetup[table]{belowskip=2pt}
\captionsetup[figure]{aboveskip=5pt}
\captionsetup[figure]{belowskip=0pt}
\captionsetup{font=small}

%%%%%%%%% PAPER ID  - PLEASE UPDATE
\def\cvprPaperID{*****} % *** Enter the CVPR Paper ID here
\def\confName{CVPR}
\def\confYear{2022}


% utils
\newcommand{\correct}[1]{\textcolor{blue}{#1 \checkmark}}
\newcommand{\wrong}[1]{\textcolor{red}{#1 $\times$}}

\begin{document}

%%%%%%%%% TITLE - PLEASE UPDATE
\title{LaTex Paper Writing Tips for Beginners}

\author{Guanying Chen \\
{\tt\small guanying2018@gmail.com}
% For a paper whose authors are all at the same institution,
% omit the following lines up until the closing ``}''.
% Additional authors and addresses can be added with ``\and'',
% just like the second author.
% To save space, use either the email address or home page, not both
}
\maketitle
{
    \hypersetup{linkcolor=black}
    \tableofcontents
}


%%%%%%%%% ABSTRACT

%%%%%%%%% BODY TEXT
\vspace{6em}
\section{Introduction}
\label{sec:intro}

\section{Common Mistakes}
\begin{itemize}
    \item There should be a space before the open parentheses: [\wrong{hello(word)}, \correct{hello (word)}].
    \item citation
    \item There should be no space before the punctuation (\eg, ,. ').
    \item \ie, \eg
    \item Do not add citations in the abstract
    \item Every equation should end with a punctuation.
    \item The first character should be capitalized
    \item keep the same tense
    \item
    \item
    \item
    \item
    \item
    \item
    \item
\end{itemize}

\section{Suggestions}
\begin{itemize}
    \item use newcommand for abbr
    \item Fig/Tab.
    \item use newcommand in equation
    \item Principal: be compact and clear.
    \item Contributions
\end{itemize}

\clearpage
\section{Sample layout for figures}
\begin{itemize}
    \item The caption should be at the bottom of the figure. 
\end{itemize}


\clearpage
\section{Sample layout for tables}
Suggestions.
\begin{itemize}
    \item The caption should be at the top of the table.
    \item The text should not be too small.
    \item 
\end{itemize}

% single column
\begin{table}[thb]\centering
    \caption{A simple table with a header row.}
    \label{tab:table1}
    \resizebox{0.48\textwidth}{!}{
    \large
    \begin{tabular}{*{10}{c}}
        \toprule
       Data & Size &  2-Exp & 3-Exp &  4-Exp & 5-Exp &  6-Exp &  7-Exp \\
        \midrule
        A & $1280\times 720$ & 1 & 2 & 3 & 4 & 5 & 4 \\
        B & $1280\times 720$ & 1 & 2 & 3 & 4 & 5 & 4 \\
        Ours & $4096\times 2168$ & 2 & 3 & 4 & 6 & 5 & 4 \\
        \bottomrule
    \end{tabular}
    }
\end{table}

\begin{table}[thb]\centering
    \caption{A table with multi-column headers.}
    \label{tab:table2}
    \resizebox{0.48\textwidth}{!}{
    \large
    \begin{tabular}{*{10}{c}}
        \toprule
        &  &  \multicolumn{2}{c}{$6-9$ frames} & \multicolumn{2}{c}{$5-7$ frames} & \multicolumn{2}{c}{$50-200$ frames} \\
       Data & Size &  2-Exp & 3-Exp &  2-Exp & 3-Exp &  2-Exp & 3-Exp \\
        \midrule
        A & $1280\times 720$ & 1 & 2 & 3 & 4 & 5 & 4 \\
        B & $1280\times 720$ & 1 & 2 & 3 & 4 & 5 & 4 \\
        Ours & $4096\times 2168$ & 2 & 3 & 4 & 6 & 5 & 4 \\
        \bottomrule
    \end{tabular}
    }
\end{table}


% require package makecell 
\begin{table}[thb]\centering
    \caption{A table with line break in the header. Line break is useful if the item name is too long.}
    \label{tab:table3}
    \resizebox{0.48\textwidth}{!}{
    \large
    \begin{tabular}{*{10}{c}}
        \toprule
         %&  &  \multicolumn{3}{c}{\makecell{Static Scenes \\ w/ Ground Truth}} & \multicolumn{3}{c}{\makecell{Dynamic Scenes \\ w/o Ground Truth}} \\
        &  &  \multicolumn{3}{c}{$6-9$ frames} & \multicolumn{3}{c}{$5-7$ frames} \\
        Data & Size &  \makecell{2-Exp \\ Scenes} &  \makecell{2-Exp \\ Scenes} &  \makecell{2-Exp \\ Scenes} &  \makecell{2-Exp \\ Scenes} &  \makecell{2-Exp \\ Scenes} &  \makecell{2-Exp \\ Scenes} \\
        \midrule
        A & $1280\times 720$ & 1 & 2 & 3 & 4 & 5 & 4 \\
        Ours & $4096\times 2168$ & 2 & 3 & 4 & 6 & 5 & 4 \\
        \bottomrule
    \end{tabular}
    }
\end{table}



\input{tables/table4}
% require package makecell 
\begin{table}[thb]\centering
    \caption{A table with multi-column headers and bold font highlights.}
    \label{tab:table5}
    \resizebox{0.48\textwidth}{!}{
    \large
    \begin{tabular}{c|c|ccc|ccc}
        \toprule
         %&  &  \multicolumn{3}{c}{\makecell{Static Scenes \\ w/ Ground Truth}} & \multicolumn{3}{c}{\makecell{Dynamic Scenes \\ w/o Ground Truth}} \\
        &  &  \multicolumn{3}{c|}{$6-9$ frames} & \multicolumn{3}{c}{$5-7$ frames} \\
        Data & Size &  \makecell{2-Exp \\ Break} &  \makecell{2-Exp \\ Break} &  \makecell{2-Exp \\ Break} &  \makecell{2-Exp \\ Break} &  \makecell{2-Exp \\ Break} &  \makecell{2-Exp \\ Break} \\
        \midrule
        A & $1280\times 720$ & 1 & 2 & 3 & 4 & \textbf{5} & \textbf{7} \\
        Ours & $4096\times 2168$ & \textbf{2} & \textbf{3} & \textbf{4} & \textbf{6} & \textbf{5} & 4 \\
        \bottomrule
    \end{tabular}
    }
\end{table}



% \usepackage{colortbl}  
% \usepackage{xcolor}

\begin{table}[thb]\centering
    \caption{A table with parallel lines for grouping and color highlight.} %
    \label{tab:ablation_study}
    \resizebox{0.48\textwidth}{!}{
    \Large
    \begin{tabular}{*{2}{l}||*{2}{c}||*{2}{c}||*{2}{c}}
        \toprule
        \multicolumn{1}{c}{} & \multicolumn{1}{c||}{} & \multicolumn{2}{c||}{Synthetic Dataset} & \multicolumn{2}{c||}{Static Data} & \multicolumn{2}{c}{Dynamicgt Data} \\
        ID & Method  & PSNRT & VDP & PSNRT & VDP & PSNRT & VDP \\
        \midrule
        0 & ANet  & 39.25 & 70.81 & 40.62 & 74.51 & 44.43 & 77.74 \\ %
        1 & BNet  & 39.69 & 70.95 & 37.61 & 75.30 & 43.70 & 78.97 \\
        \rowcolor{gray!20}
        2 & ANet + BNet  & \textbf{40.34} & \textbf{71.79} & \textbf{41.18} & \textbf{76.15} & \textbf{45.46} & \textbf{79.09} \\
        \midrule
        3 & ANet + BNet w/o C & 39.72 & 71.38 & 40.52 & 74.79 & 45.09 & 78.24 \\ %
        4 & ANet + BNet w/o D & 40.03 & 71.66 & 40.80 & 76.12 & 45.17 & 78.99 \\ %
        \bottomrule
    \end{tabular}
    }
\end{table}

\begin{table}[h]
    \centering
    \caption{A table for illustrating the network architecture.}
    \label{tab:refineNet}
	\resizebox{0.48\textwidth}{!}{
        \small
        \begin{tabular}[t]{|*{6}{l|}}
        \hline
        \multicolumn{6}{|c|}{\textbf{RefineNet}} \\
        \hline
        \textbf{layer} & \textbf{k} & \textbf{s} & \textbf{chns} & \textbf{d-f} & \textbf{input} \\
        \hline 
        conv1   & 9     & 1    & 8/64   & 1    & Image+m\_1+a\_1+f\_1\\
        conv2   & 4     & 2    & 64/64  & 2    & conv1 \\
        conv3   & 4     & 2    & 64/64  & 4    & conv2 \\
        conv4   & 4     & 2    & 64/64  & 8    & conv3 \\
        \hline 
        ResBlock1 & 3     & 1  & 64/64  & 8    & conv4 \\
        ResBlock2 & 3     & 1  & 64/64  & 8    & ResBlock1 \\
        ResBlock3 & 3     & 1  & 64/64  & 8    & ResBlock2 \\
        ResBlock4 & 3     & 1  & 64/64  & 8    & ResBlock3 \\
        ResBlock5 & 3     & 1  & 64/64  & 8    & ResBlock4 \\
        \hline 
        deconv1\_a  & 4   & 2    & 64/64 & 4   & ResBlock5 \\
        deconv2\_a  & 4   & 2    & 64/64 & 2   & deconv1\_a \\
        deconv3\_a  & 4   & 2    & 64/64 & 1   & deconv2\_a \\
        \rowcolor{gray!25}
        a\_refined  & 3   & 1    & 65/1 & 1   & deconv3\_a+a\_1\\
        \hline
        deconv1\_f  & 4   & 2    & 64/64 & 4   & ResBlock5 \\
        deconv2\_f  & 4   & 2    & 64/64 & 2   & deconv1\_f \\
        deconv3\_f  & 4   & 2    & 64/64 & 1   & deconv2\_f \\
        \rowcolor{gray!25}
        f\_refined        & 3   & 1    & 66/2 & 1   & deconv3\_f+f\_1\\
        \hline
        \end{tabular}
    }
\end{table}


\begin{table}[thb] \centering
    \caption{An example for table with images.} 
    \label{tab:res_synth_mirco_baseline}
    \includegraphics[width=0.088\textwidth,height=0.085\textwidth]{example-image-a}
    \includegraphics[width=0.088\textwidth,height=0.085\textwidth]{example-image-b}
    \includegraphics[width=0.088\textwidth,height=0.085\textwidth]{example-image-c}
    \raisebox{0.9\height}{
    \resizebox{0.19\textwidth}{!}{
    \large
    \begin{tabular}{ccc}
        \toprule
        Type & Range & MAE \\
        \midrule
        (a) & 144$\times$144 & 4.21 \\ 
        (b) & 37$\times$37 & 10.90 \\
        (c) & 22$\times$22 & 18.72 \\
        \bottomrule
    \end{tabular}
    }}
    \\
    \vspace{-0.2em}
    \makebox[0.09\textwidth]{\footnotesize (a)} 
    \makebox[0.09\textwidth]{\footnotesize (b)}
    \makebox[0.09\textwidth]{\footnotesize (c)}
    \makebox[0.19\textwidth]{\footnotesize (d) Normal estimation}
    \\
\end{table}


% two column
\input{tables/table8}
\input{tables/table9}
\newcommand{\PSNRT}{PSNR\xspace}
\newcommand{\VQM}{HDR-VQM\xspace}
\newcommand{\VDP}{HDR-VDP2\xspace}
\newcommand{\LowExposure}{Low-Exposure\xspace}
\newcommand{\MiddleExposure}{Middle-Exposure\xspace}
\newcommand{\HighExposure}{High-Exposure\xspace}
\newcommand{\AllExposure}{All-Exposure\xspace}

\newcommand{\Frst}[1]{\textcolor{red}{\textbf{#1}}}
\newcommand{\Scnd}[1]{\textcolor{blue}{\textbf{#1}}}

\begin{table*}[t] \centering
    \caption{A two-column table with two sub-tables. \Frst{Red} text indicates the best and \Scnd{blue} text indicates the second best result, respectively.}
    \label{tab:res_real_data_wide}
    \makebox[\textwidth]{\small (a) Results on dataset A.} 
    \resizebox{\textwidth}{!}{
        \Large
    \begin{tabular}{l||*{2}{c}|*{2}{c}|*{3}{c}||*{2}{c}|*{2}{c}|*{2}{c}|*{3}{c}}
        \toprule
        & \multicolumn{7}{c||}{2-Exposure} & \multicolumn{9}{c}{3-Exposure} \\
        & \multicolumn{2}{c}{\LowExposure} & \multicolumn{2}{c}{\HighExposure} & \multicolumn{3}{c||}{\AllExposure} & \multicolumn{2}{c}{\LowExposure} & \multicolumn{2}{c}{\MiddleExposure} & \multicolumn{2}{c}{\HighExposure} & \multicolumn{3}{c}{\AllExposure} \\
        Method & \PSNRT & \VDP & \PSNRT & \VDP  & \PSNRT & \VDP & \VQM & \PSNRT & \VDP & \PSNRT & \VDP  & \PSNRT & \VDP & \PSNRT & \VDP & \VQM \\
        \midrule
        Method A & \Scnd{40.00} & 73.70 & \Scnd{40.04} & \Scnd{70.08} & \Scnd{40.02} & 71.89 & \Scnd{76.22} 
                      & \Scnd{39.61} & 73.24 & \Frst{39.67} & \Frst{73.24} & \Frst{40.01} & 67.90 & \Frst{39.77} & 70.37 & 79.55 \\
        Method B      & 34.54 & 80.22 & 39.25 & 65.96 & 36.90 & 73.09 & 65.33 
                      & 36.51 & 77.78 & 37.45 & 69.79 & 39.02 & 64.57 & 37.66 & 70.71 & 70.13 \\
        Method C  & 39.79 & \Scnd{81.02} & 39.96 & 67.25 & 39.88 & \Scnd{74.13} & 73.84 
                      & 39.48 & \Scnd{78.13} & 38.43 & 70.08 & 39.60 & \Scnd{67.94} & 39.17 & \Scnd{72.05} & \Scnd{80.70} \\
        Ours          & \Frst{41.95} & \Frst{81.03} & \Frst{40.41} & \Frst{71.27} & \Frst{41.18} & \Frst{76.15} & \Frst{78.84}
                      & \Frst{40.00} & \Frst{78.66} & \Scnd{39.27} & \Scnd{73.10} & \Scnd{39.99} & \Frst{69.99} & \Scnd{39.75} & \Frst{73.92} & \Frst{82.87} \\
        \bottomrule
    \end{tabular}
    }

    \vspace{0.5em}
    \makebox[\textwidth]{\small (b) Results on dataset B.} 
    \resizebox{\textwidth}{!}{
        \Large
    \begin{tabular}{l||*{2}{c}|*{2}{c}|*{3}{c}||*{2}{c}|*{2}{c}|*{2}{c}|*{3}{c}}
        \toprule
        & \multicolumn{7}{c||}{2-Exposure} & \multicolumn{9}{c}{3-Exposure} \\
        & \multicolumn{2}{c}{\LowExposure} & \multicolumn{2}{c}{\HighExposure} & \multicolumn{3}{c||}{\AllExposure} & \multicolumn{2}{c}{\LowExposure} & \multicolumn{2}{c}{\MiddleExposure} & \multicolumn{2}{c}{\HighExposure} & \multicolumn{3}{c}{\AllExposure} \\
        Method & \PSNRT & \VDP & \PSNRT & \VDP  & \PSNRT & \VDP & \VQM & \PSNRT & \VDP & \PSNRT & \VDP  & \PSNRT & \VDP & \PSNRT & \VDP & \VQM \\
        \midrule
        Method A & 37.73 & 74.05 & 45.71 & 66.67 & 41.72 & 70.36 & 85.33 
                      & 37.53 & 72.03 & 36.38 & 65.37 & 34.73 & 62.24 & 36.21 & 66.55 & 84.43 \\
        Method B      & 36.41 & 85.68 & \Scnd{49.89} & \Scnd{69.90} & 43.15 & 77.79 & 78.92 
                      & 36.43 & 77.74 & 39.80 & \Scnd{67.88} & \Scnd{43.03} & \Scnd{64.74} & 39.75 & \Scnd{70.12} & 87.93 \\
        Method C  & \Scnd{39.94} & \Scnd{86.77} & 49.49 & 69.04 & \Scnd{44.72} & \Scnd{77.91} & \Scnd{87.16} 
                      & \Scnd{38.34} & \Scnd{78.04} & \Scnd{41.21} & 66.07 & 42.66 & 64.01 & \Scnd{40.74} & 69.37 & \Scnd{89.36}\\
        Ours          & \Frst{40.83} & \Frst{86.84} & \Frst{50.10} & \Frst{71.33} & \Frst{45.46} & \Frst{79.09} & \Frst{87.40} 
                      & \Frst{38.77} & \Frst{78.11} & \Frst{41.47} & \Frst{68.49} & \Frst{43.24} & \Frst{65.08} & \Frst{41.16} & \Frst{70.56} & \Frst{89.56}\\
        \bottomrule
    \end{tabular}
    }
\end{table*}

\begin{table*}[t] \centering
    \caption{A two-column table with images in the header. Images are useful to visualize each item in the header.} 
    \label{tab:table10}
    \resizebox{\textwidth}{!}{%
        %\large
        \begin{tabular}{l|*{6}{cc}cc}
            \toprule
            & \multicolumn{2}{c}{\includegraphics[height=0.09\textwidth]{example-image}}
            & \multicolumn{2}{c}{\includegraphics[height=0.09\textwidth]{example-image}}
            & \multicolumn{2}{c}{\includegraphics[height=0.09\textwidth]{example-image}}
            & \multicolumn{2}{c}{\includegraphics[height=0.09\textwidth]{example-image}}
            & \multicolumn{2}{c}{\includegraphics[height=0.09\textwidth]{example-image}}
            & \multicolumn{2}{c}{\includegraphics[height=0.09\textwidth]{example-image}}
            %\vspace{-0.5em}
            \\
            & \multicolumn{2}{c}{\makebox[0.18\textwidth]{\emph{Helmet Side}}}
            & \multicolumn{2}{c}{\makebox[0.18\textwidth]{\emph{Plant}}}
            & \multicolumn{2}{c}{\makebox[0.18\textwidth]{\emph{Fighting Knight}}}
            & \multicolumn{2}{c}{\makebox[0.18\textwidth]{\emph{Kneeling Knight}}}
            & \multicolumn{2}{c}{\makebox[0.18\textwidth]{\emph{Standing Knight}}}
            & \multicolumn{2}{c}{\makebox[0.18\textwidth]{\emph{Helmet Front}}}
            & \multicolumn{2}{c}{\makebox[0.18\textwidth]{average}} 
            \\
            \makebox[0.12\textwidth]{model\hfill}  & dir.    & int.  &  dir.    & int.  &  dir.     & int. &  dir.    & int. & dir.    & int.  &  dir.  & int.  &  dir.  & int. \\
            \midrule
            Method A & 25.40   & 0.576     & 20.56     & 0.227    & 69.50    & 1.137     & 46.69    & 9.805     & 33.81    & 1.311     & 81.60    & \textbf{0.133} & 46.26   & 2.198    \\
            Method B & 6.57	& 0.212	    & 16.06	    & 0.170	   & 15.95 	  & 0.214	  & 19.84	 & 0.199	 & 11.60    & 0.286	    & 11.62	   & 0.248	  & 13.61    & 0.221      \\
            Method C & \textbf{5.33}& \textbf{0.096} & \textbf{10.49} & \textbf{0.154}& \textbf{13.42}& \textbf{0.168} & \textbf{14.41}& \textbf{0.181} & \textbf{5.31} & \textbf{0.198} & \textbf{6.22} & 0.183     & \textbf{9.20} & \textbf{0.163} \\
            \bottomrule
        \end{tabular}
    }
\end{table*}


\begin{table*}[thb]
    \centering
    \caption{A two-column table for illustrating the network architecture.}
    \label{tab:table_network1}
	\resizebox{0.90\textwidth}{!}{
    \begin{tabular}{cc}
        \begin{tabular}[t]{|*{6}{l|}}
        \hline
        \multicolumn{6}{|c|}{\textbf{Encoder}} \\
        \hline
        \textbf{layer} & \textbf{k} & \textbf{s} & \textbf{chns} & \textbf{d-f} & \textbf{input} \\
        \hline 
        conv1   & 3     & 1     & 3/16    & 1    & Image \\
        conv1b  & 3     & 1     & 16/16   & 1    & conv1 \\ 
        conv2   & 3     & 2     & 16/16   & 2    & conv1b \\
        conv2b  & 3     & 1     & 16/16   & 2    & conv2 \\ 
        conv3   & 3     & 2     & 16/32   & 4    & conv2b \\
        conv3b  & 3     & 1     & 32/32   & 4    & conv3 \\ 
        conv4   & 3     & 2     & 32/64   & 8    & conv3b \\
        conv4b  & 3     & 1     & 64/64   & 8    & conv4 \\ 
        conv5   & 3     & 2     & 64/128  & 16   & conv4b \\
        conv5b  & 3     & 1     & 128/128 & 16   & conv5 \\ 
        conv6   & 3     & 2     & 128/256 & 32   & conv5b \\
        conv6b  & 3     & 1     & 256/256 & 32   & conv6 \\ 
        conv7   & 3     & 2     & 256/256 & 64   & conv6b \\
        conv7b  & 3     & 1     & 256/256 & 64   & conv7 \\
        \hline
        \end{tabular}
        \begin{tabular}[t]{|*{6}{l|}}
            \hline
            \multicolumn{6}{|c|}{\textbf{Decoder}} \\
            \hline
            \textbf{layer} & \textbf{k} & \textbf{s} & \textbf{chns} & \textbf{d-f} & \textbf{input} \\
            \hline 
            conv\_up7\_m & 3     & 1     & 256/256   & 32  & conv7b \\
            conv\_up7\_a & 3     & 1     & 256/256   & 32  & conv7b \\
            conv\_up7\_f & 3     & 1     & 256/256   & 32  & conv7b \\
            \hline 
            \multicolumn{6}{|c|}{conv\_up7=conv\_up7\_m+conv\_up7\_a+conv\_up7\_f} \\
            \hline
            conv\_up6\_m & 3     & 1     & 256/128   & 16  & conv\_up7+conv6b\\
            conv\_up6\_a & 3     & 1     & 256/128   & 16  & conv\_up7+conv6b\\
            conv\_up6\_f & 3     & 1     & 256/128   & 16  & conv\_up7+conv6b\\
            \hline 
            \multicolumn{6}{|c|}{conv\_up6=conv\_up6\_m+conv\_up6\_a+conv\_up6\_f} \\
            \hline
            conv\_up5\_m & 3     & 1     & 128/64    & 8   & conv\_up6+conv5b \\
            conv\_up5\_a & 3     & 1     & 128/64    & 8   & conv\_up6+conv5b \\
            conv\_up5\_f & 3     & 1     & 128/64    & 8   & conv\_up6+conv5b \\
            \hline
            \multicolumn{6}{|c|}{conv\_up5=conv\_up5\_m+conv\_up5\_a+conv\_up5\_f} \\
            \hline
            m\_4         & 3     & 1     & 128/2     & 8   & conv\_up5+conv4b \\
            a\_4         & 3     & 1     & 128/1     & 8   & conv\_up5+conv4b \\
            f\_4         & 3     & 1     & 128/2     & 8   & conv\_up5+conv4b \\
            conv\_up4\_m & 3     & 1     & 128/32    & 4   & conv\_up5+conv4b \\
            conv\_up4\_a & 3     & 1     & 128/32    & 4   & conv\_up5+conv4b \\
            conv\_up4\_f & 3     & 1     & 128/32    & 4   & conv\_up5+conv4b \\
            \hline
            \multicolumn{6}{|c|}{conv\_up4=conv\_up4\_m+conv\_up4\_a+conv\_up4\_f} \\
            \hline
            m\_3         & 3     & 1     & 69/2      & 4   & conv\_up4+conv3b+(m\_4$^{\times 2}$+a\_4$^{\times 2}$+a\_4$^{\times 2}$) \\
            a\_3         & 3     & 1     & 69/1      & 4   & conv\_up4+conv3b+(m\_4$^{\times 2}$+a\_4$^{\times 2}$+a\_4$^{\times 2}$) \\
            f\_3         & 3     & 1     & 69/2      & 4   & conv\_up4+conv3b+(m\_4$^{\times 2}$+a\_4$^{\times 2}$+a\_4$^{\times 2}$) \\
            conv\_up3\_m & 3     & 1     & 69/16     & 2   & conv\_up4+conv3b+(m\_4$^{\times 2}$+a\_4$^{\times 2}$+a\_4$^{\times 2}$) \\
            conv\_up3\_a & 3     & 1     & 69/16     & 2   & conv\_up4+conv3b+(m\_4$^{\times 2}$+a\_4$^{\times 2}$+a\_4$^{\times 2}$) \\
            conv\_up3\_f & 3     & 1     & 69/16     & 2   & conv\_up4+conv3b+(m\_4$^{\times 2}$+a\_4$^{\times 2}$+a\_4$^{\times 2}$) \\
            \hline
            \multicolumn{6}{|c|}{conv\_up3=conv\_up3\_m+conv\_up3\_a+conv\_up3\_f} \\
            \hline
            m\_2         & 3     & 1     & 37/2      & 2   & conv\_up3+conv2b+(m\_3$^{\times 2}$+a\_3$^{\times 2}$+a\_3$^{\times 2}$) \\
            a\_2         & 3     & 1     & 37/1      & 2   & conv\_up3+conv2b+(m\_3$^{\times 2}$+a\_3$^{\times 2}$+a\_3$^{\times 2}$) \\
            f\_2         & 3     & 1     & 37/2      & 2   & conv\_up3+conv2b+(m\_3$^{\times 2}$+a\_3$^{\times 2}$+a\_3$^{\times 2}$) \\
            conv\_up2\_m & 3     & 1     & 37/16     & 1   & conv\_up3+conv2b+(m\_3$^{\times 2}$+a\_3$^{\times 2}$+a\_3$^{\times 2}$) \\
            conv\_up2\_a & 3     & 1     & 37/16     & 1   & conv\_up3+conv2b+(m\_3$^{\times 2}$+a\_3$^{\times 2}$+a\_3$^{\times 2}$) \\
            conv\_up2\_f & 3     & 1     & 37/16     & 1   & conv\_up3+conv2b+(m\_3$^{\times 2}$+a\_3$^{\times 2}$+a\_3$^{\times 2}$) \\
            \hline
            \multicolumn{6}{|c|}{conv\_up2=conv\_up2\_m+conv\_up2\_a+conv\_up2\_f} \\
            \hline
            m\_1         & 3     & 1     & 37/2      & 1   & conv\_up2+conv1b+(m\_2$^{\times 2}$+a\_2$^{\times 2}$+a\_2$^{\times 2}$) \\
            a\_1         & 3     & 1     & 37/1      & 1   & conv\_up2+conv1b+(m\_2$^{\times 2}$+a\_2$^{\times 2}$+a\_2$^{\times 2}$) \\
            f\_1         & 3     & 1     & 37/2      & 1   & conv\_up2+conv1b+(m\_2$^{\times 2}$+a\_2$^{\times 2}$+a\_2$^{\times 2}$) \\
            \hline                                                                         
        \end{tabular}
    \end{tabular}
    }
\end{table*}


 
\clearpage
%%%%%%%%% REFERENCES
{\small
\bibliographystyle{ieee_fullname}
\bibliography{egbib}
}

\end{document}
